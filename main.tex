\documentclass[a4paper, openany, 12pt]{article}

\usepackage{amsthm}
\usepackage{longtable}
\usepackage{comment}
\usepackage{amsmath}
\usepackage{placeins}
\newtheorem{theorem}{Theorem}

%% подключаем стандарт библиографии
\bibliographystyle{gost71u}

%% для "Abstract" в классе book
% \newenvironment{abstract}{}{}
% \usepackage{abstract}

%% подключаем преамбулу: в ней содержится подключение всех необходимых пакетов
\input{include/preambule.tex}

\begin{document}
    %% титульник
    \input{include/title.tex}
    %% аннотоция
    \begin{abstract}

    \begin{center}
        \large{Аугментации в задаче распознавания рукописного текста} \\
    \large\textit{Курочкин Павел Сергеевич} \\[1 cm]

    В современных методах распознавания рукописного текста ключевым элементом является наличие обширного массива аннотированных данных. Однако, в случае ограниченности данных, эффективное применение методов аугментации становится критически важным для повышения точности моделей. Manifold Mixup \hyperlink{cite.Ver18}{[1]} - современный метод аугментации данных, он представляет собой подход, в котором объединяются два или более изображения или их признаковые карты для создания новых образцов данных. В данном исследовании предлагается метод адаптации Manifold Mixup для использования с подходами, основанными на Connectionist Temporal Classification \hyperlink{cite.Gra06}{[2]}, в контексте задачи распознавания рукописного текста. Проводится всестороннее исследование данного метода, анализируется его влияние на процесс обучения нейронной сети. Результаты исследования показывают значительное улучшение результатов распознавания текста при использовании Manifold Mixup.

    \vfill

    \textbf{Abstract} \\[1 cm]

    Augmentations in handwritten text recognition
    \end{center}

\end{abstract}
\newpage
    %% содержание
    \tableofcontents{}
    \newpage

    % \fontsize{14}{16}\selectfont
    \section{Введение}
\label{sec:Chapter0} \index{Chapter0}

Распознавание текста является важным этапом в большинстве приложений по анализу изображений документов. Оно позволяет автоматически получать доступ к информации, содержащейся на страницах. За последнее десятилетие произошло огромное улучшение систем распознавания рукописного текста, связанное с появлением новых подходов и архитектур. Совершенствование рекурентных сетей (в частности, одними из современных подходов являются 2D-LSTM \hyperlink{cite.Gra08}{[4]}, \hyperlink{cite.Bas14}{[5]} и 1D-LSTM \hyperlink{cite.Joa17}{[6]}, \hyperlink{cite.The17}{[7]}), а также появление архитектуры Transformer \hyperlink{cite.Vas17}{[3]} оказали значительное влияние на этот прогресс.
\subsection{Проблема}

Использование современных методов нейронных сетей помогает решить проблему высокой стилистической гетерогенности рукописного текста. Однако для таких мощных моделей с большим количеством параметров требуется значительное количество аннотированных изображений для обучения. Было предложено несколько методов, позволяющих уменьшить необходимость аннотированных данных при обучении систем распознавания текста.

Во-первых, обучающий набор можно расширить, добавив синтетические изображения. Это можно сделать, используя рукописные шрифты \hyperlink{cite.Mur03}{[9]} или создавая изображения текстовых строк на основе индивидуально извлеченных изображений реальных букв \hyperlink{cite.She16}{[10]}. Кроме того, для создания рукописных текстов можно применять генеративно-состязательные сети \hyperlink{cite.Alo19}{[11]}. 

Также, при ограниченном объеме данных, необходимо бороться с проблемой переобучения. Для этого было предложено множество подходов к регуляризации, таких как weight decay, dropout \hyperlink{cite.Sut14}{[12]}, batch normalization \hyperlink{cite.Iof15}{[21]} или раннее прекращение обучения, которые могут быть использованы в процессе обучения сети.

Наконец, для увеличения разнообразия обучающего набора, улучшения обобщающей способности модели и снижения риска переобучения широко применяются методы аугментации данных. Изображения могут быть отражены горизонтально и вертикально, подвергнуты поворотам и угловым трансформациям, изменены по размеру и масштабу. Также возможно добавление шума, обрезка изображений, цветовые трансформации и искажения в различных формах \hyperlink{cite.Cur17}{[13]}, \hyperlink{cite.Bas14}{[5]}. Подробнее методы аугментации для задачи распознавания рукописного текста описаны в \hyperref[sec:Chapter2]{Глава 3}.

В \hyperlink{cite.Cha11}{[14]} для создания новых образцов было предложено интерполировать признаки образцов одного класса. В \hyperlink{cite.Dev17}{[15]} было предложено использовать данный подход для признаков образцов на выходе промежуточного слоя нейронной сети. В \hyperlink{cite.Ver18}{[1]} объединяются обе эти идеи и предлагается смешивать целевую переменную и изображения из разных классов, или их промежуточные признаки на различных уровнях сети. Последний подход был назван Manifold Mixup. В \hyperlink{cite.Bas19}{[16]} представлен способ адаптации Manifold Mixup для работы с решениями, основанными на Connectionist Temporal Classification \hyperlink{cite.Gra06}{[2]}.

Цель данного исследования заключается в анализе методов аугментации изображений в контексте распознавания рукописного текста и более детальном рассмотрении различных аспектов метода Manifold Mixup. Выбор использования Manifold Mixup обоснован следующими уникальными характеристиками данного метода:
\begin{enumerate}
\item Его легкость в обобщении на широкий спектр задач компьютерного зрения, решаемых с помощью нейронных сетей.
\item Согласно результатам исследований \hyperlink{cite.Ver18}{[1]}, \hyperlink{cite.Bas19}{[16]}, данный метод значительно способствует качеству и обобщенности обученной модели.
\end{enumerate}
 
\subsection{Постановка задачи}
Рассматриваемые методы аугментации изображений, включая метод Manifold Mixup \hyperlink{cite.Ver18}{[1]}, исследуемый в данной работе, предназначены для преодоления следующих проблем, возникающих в задаче распознавания рукописного текста:
\begin{enumerate}
\item Разнообразие изображений рукописного текста из-за различий в стилях авторов и фонах документов.
\item Недостаток аннотированных данных, что затрудняет обобщение моделей.
\item Риск переобучения из-за упомянутых выше проблем и большого количества параметров в сети.
\end{enumerate}
В данном исследовании проводится анализ различных методов агументаций, решающих данные проблемы. А также исследуются следующие аспекты метода Manifold Mixup:
\begin{enumerate}
\item Методы формирования батчей из изображений различных длин и их влияние на стабильность процесса обучения.
\item Роль выбора промежуточного слоя в процессе обучения для различных архитектур.
\item Воздействие Manifold Mixup на процесс обучения в зависимости от объема имеющихся данных.
\end{enumerate}

В \hyperref[sec:Chapter1]{Главe 2} приводится обзор существующих архитектур для решения задачи распознавания рукописного текста. Также представлена архитектура, применяемая в данном исследовании, обосновывается выбор данной архитектуры.


\hyperref[sec:Chapter2]{Глава 3} посвящена обзору существующих методов аугментации изображений для задачи распознавания рукописного текста, включая Manifold Mixup. Оценивается эффективность их применения для решения вышеупомянутых задач.

В \hyperref[sec:Chapter3]{Главе 4} и \hyperref[sec:Chapter4]{Главе 5} представлено описание практической части: конфигурация и эксперименты соответсвенно.

И, наконец, в \hyperref[sec:Chapter5]{Главе 6} содержатся выводы из проведенного исследования.

\newpage %% fsd
    \section{Архитектура решения}
\label{sec:Chapter1} \index{Chapter1}

\begin{figure}
    \centering
    \includegraphics[scale=1]{./images/activation-funs-grad.png}
    \caption{\protect\hypertarget{image1}{Градиенты некоторых популярных функций активации.}}
\end{figure}


Современные модели, применяемые в задаче распознавания рукописного текста, обычно включают в себя три ключевых компонента:
\begin{enumerate}
\item Сверточную нейронную сеть, используемую для извлечения признаков из входных данных.
\item Последовательный энкодер
\item Декодер, который завершает процесс, трансформируя закодированные данные в окончательную текстовую транскрипцию с учетом информации, полученной от энкодера.
\end{enumerate}

\subsection{Извлечение визуальных признаков с помощью сверточной сети}
Подобно другим задачам компьютерного зрения, сверточная сеть используется для извлечения соответствующих визуальных признаков из текстовых изображений. Следующие архитектуры являются популярным выбором: ResNet \hyperlink{cite.Kai15}{[17]}, Inception \hyperlink{cite.Sze14}{[18]}, MobileNet \hyperlink{cite.San18}{[19]}. Увеличение сложности сверточной сети, как правило, приводит к умеренному приросту точности модели \hyperlink{cite.Her21}{[20]}. В этой работе используется архитектура ResNet. 

\newpage

\begin{figure}
    \centering
    \includegraphics[scale=1]{./images/ResBlock.png}
    \caption{\protect\hypertarget{image2}{Residual блок.}}
\end{figure}

\subsubsection{Residual connection}
При обучении глубоких моделей градиент имеет тенденцию становиться очень маленьким: это называется проблема исчезающего градиента. Это связано с тем, что градиент проходит через ряд слоев, каждый из которых может уменьшить его. Например, градиент многих популярных функций активации приближается к нулю на значительной части числовой прямой \hyperlink{image1}{[Рис 1]}. 

Одним из решений проблемы исчезающего градиента является использование в качестве слоев нейронной сети residual блока \hyperlink{image2 }{[Рис 2]}, определенного следующим образом:
\begin{equation}
	\mathcal{F}_{l}^{'}(x) = \mathcal{F}_{l}(x) + x
\end{equation}
где $\mathcal{F}_{l}$ - стандартное нелинейное отображение (например, линейное - функция активации - линейное). Зачастую легче научиться предсказывать небольшое возмущение на входе, чем результат напрямую.

Модель с residual блоком имеет такое же количество параметров, как и модель без него, но ее легче тренировать. Причина в том, что градиенты могут перетекать непосредственно из выходных данных в более ранние слои. Чтобы убедиться в этом, рассмотрим градиент функции ошибки по параметрам слоя $l$. Имеем
\begin{equation}
	z_L = z_l + \sum_{i=l}^{L-1} \mathcal{F}_i(z_i ; \theta_i)
\end{equation}
где $z_i$ - признаки на выходе из $i$-го слоя сети. Таким образом, мы можем вычислить градиент функции потерь относительно параметров $l$-го слоя следующим образом:
\begin{equation}
	\frac{\partial \mathcal{L}}{\partial \theta_{l}} = 
	\frac{\partial z_{l}}{\partial \theta_{l}} \frac{\partial \mathcal{L}}{\partial z_{L}}
	(1 + \sum_{i=l}^{L-1} \frac{\partial \mathcal{F}_i(z_i; \theta_i)}{\partial z_{l}})
\end{equation}
Таким образом, мы видим, что градиент на слое $l$ напрямую зависит от градиента на слое $L$, причем независимо от глубины сети.
\begin{figure}
    \centering
    \includegraphics[scale=0.3]{./images/ResNet18.png}
    \caption{\protect\hypertarget{image3}{Архитектура ResNet-18.}}
\end{figure}

\newpage
\subsubsection{ResNet}
Победителем конкурса по компьютерному зрению \href{https://image-net.org/challenges/LSVRC/2015/}{ImageNet} 2015 года стала команда Microsoft, предложившая модель, известную сейчас как ResNet \hyperlink{image3}{[Рис 3]}. Данная модель состоит из residual блоков, имеющих следующий вид: свертка-BN-relu-свертка-BN, где BN - батч нормализация \hyperlink{cite.Iof15}{[21]}). Подобная архитектура позволяет обучать очень глубокие модели, а также решает проблему затухания градиентов. Именно из-за указанных преимуществ, а также из практического опыта, данная архитектура была отобрана для проведения данного исследования.

\begin{figure}
    \centering
    \includegraphics[scale=0.3]{./images/GRCL.png}
    \caption{\protect\hypertarget{image4}{Архитектура GRCL.}}
\end{figure}

\subsection{Последовательный энкодер}
Сверточная сеть, предназначенная для извлечения визуальных признаков, имеет ограниченное рецептивное поле, что ограничивает ее способность учитывать широкий контекст информации. Это создает сложности при обработке длинных последовательностей в сложных сценариях, таких как распознавание рукописного текста. Для учета контекста на больших расстояниях используется энкодер. Существует много различных
архитектур энкодера, далее будут рассмотрены основные из них.

\subsubsection{Рекуррентные нейронные сети}
Рекуррентная нейронная сеть — это нейронная сеть, которая отображает входное пространство последовательности в выходное пространство последовательностей с сохранением состояния. То есть элемент выходной последовательности $y_t$ зависит не только от элемента входной последовательности $x_t$, но и от скрытого состояния системы $h_t$, которое обновляется. Для простоты обозначений пусть T будет длиной вывода (с учетом того, что она выбирается динамически). Тогда рекурентная сеть соответствует следующей условной генеративной модели:

\begin{equation}
	p(y_{1:T} | x) = \sum_{h_{1:T}} p(y_{1:T}, h_{1:T} | x) = \sum_{h_{1:T}} \prod_{t=1}^{T} p(y_{t} | h_{t})p(h_{t} | h_{t-1}, y_{t-1}, x)
\end{equation}
где $h_t$ — скрытое состояние, и где мы определяем $p(h_1 |h_0, y_0, x) = p(h_1 |x)$ как начальное скрытое состояние. Мы предполагаем, что скрытое состояние вычисляется детерминированно следующим образом:

\begin{equation}
	p(h_t | h_{t-1}, y_{t-1}, x) = \mathbb{I}(h_t = f (h_{t-1}, y_{t-1}, x)) 
\end{equation}
для некоторой детерминированной функции $f$. Функция обновления $f$ обычно задается выражением

\begin{equation}
	h_t = \varphi (W_{xh} [x; y_{t-1}] + W_{hh}h_{t-1} + b_h) 
\end{equation}

Рекурентные сети с достаточным количеством скрытых единиц в принципе могут запоминать входные данные из далекого прошлого. Однако сети с "ванильной" архитектурой не могут этого сделать из-за проблемы исчезающего градиента. Существует архитектурное решение данной проблемы, в котором мы обновляем скрытое состояние аддитивным способом, аналогично residual блокам в ResNet: GRU и LSTM.

Различные варианты рекуррентных сетей использовались для решения задачи распознавания рукпоисного текста: LSTM \hyperlink{cite.Bas14}{[5]} \hyperlink{cite.Gra08}{[4]}, BiLSTM \hyperlink{cite.The17}{[6]} \hyperlink{cite.Joa17}{[7]}, Gated Recurrent Convolution Layer (GRCL) \hyperlink{image4}{[Рис 4]} \hyperlink{cite.Lux17}{[22]}.

\begin{figure}
    \centering
    \includegraphics[scale=0.25]{./images/competition.png}
    \caption{\protect\hypertarget{image5}{Сравнение различных архитектур для задачи распознавания текста. \\ Взято из \protect\hyperlink{cite.Her21}{[20]}}.}
\end{figure}

\subsubsection{Self-Attention}
Self-Attention энкодер, предложенный в \hyperlink{cite.Vas17}{[3]} широко используется в задачах из NLP и компьютерного зрения. Распознавание текстовых строк, как задача преобразования изображения в последовательность, не является исключением. Self-attention способен улавливать долгосрочные зависимости в последовательностях лучше, чем рекуррентные сети, благодаря возможности обрабатывать взаимодействия между всеми элементами последовательности одновременно. Также, в отличие от рекуррентных сетей, self-attention не сталкивается с проблемой затухания градиентов.

Выход сверточной сети, с удаленным измерением высоты изображения ($X \in \mathbb{R} ^ {n \times d}$), поступает в энкодер. Выход $Y$ слоя Attention получается следующим образом:

\begin{equation}
\begin{split}
	Q = X W_q \\
	K = X W_k \\
	V = X W_v \\
	Y = softmax(\frac{Q K^T}{\sqrt{d}}V)
\end{split}
\end{equation}
где $W_q$, $W_k$ и $W_v$ — матрицы параметров размера $d \times d$, которые задают проекцию входной последовательности $X$ в пространство запросов, ключей и значений соответственно. Закодированный признак Y представляет собой выпуклую комбинацию вычисленных значений $V$, матрица  сходства вычисляется как скалярное произведение запросов и ключей.

Эффективность "ванильного" Attention может быть низкой, поскольку данный слой инвариантен к перестановкам, и, следовательно, игнорирует порядок элементов входной последовательности. Чтобы преодолеть эту проблему, к признакам элементов входной последовательности добавлятся информация о позиции элемента (positional embedding). Можно представить positional embedding как матрицу $\mathbf{P}^{n \times d}$.

В \hyperlink{cite.Vas17}{[3]} было предложено использовать синусоидальный базис следующего вида:
\begin{equation}
	p_{i,2j} = \sin(\frac{i}{C^{\frac{2j}{d}}}), \quad p_{i,2j+1} = \cos(\frac{i}{C^{\frac{2j}{d}}}) 
\end{equation}
где $C$ соответствует некоторой максимальной длине последовательности. Одним из важных плюсов данного представления является то, что представление одной позиции линейно предсказуемо относительно любой другой, если известно их относительное расстояние. В частности, выполняется $p_{t+ \phi} = f (p_t)$, где $f$ - некоторое линейное отображение. А именно
\begin{equation}
\begin{pmatrix}
	\sin(\omega_k (t + \phi)) \\
	\cos(\omega_k (t + \phi))
\end{pmatrix} = 
\begin{pmatrix}
	\cos (\omega_k \phi)  & \sin (\omega_k \phi) \\
	-\sin (\omega_k \phi) & \cos (\omega_k \phi)
\end{pmatrix}
\begin{pmatrix}
	\sin(\omega_k t) \\
	\cos(\omega_k t)
\end{pmatrix}
\end{equation}
То есть при маленьком $\phi$ имеем $p_{t + \phi} \approx p_t$. Positional embedding, как правило, прибавляется ко входу, то есть:
$PosEmbed(X) = X + \mathbf{P}$.

Существуют также другие варианты. Например, relative positional embedding \hyperlink{cite.Sha18}{[23]}.

\subsection{Декодер}
Декодеры берут признаки закодированной последовательности и пытаются декодировать ее текстовое содержимое. Архитектуры декодеров для рекурентных сетей были описаны ранее. Два других популярных подхода к решению этой задачи: transformer \hyperlink{cite.Vas17}{[3]} и Connectionist Temporal Classification \hyperlink{cite.Gra06}{[2]} декодеры.

\subsubsection{Transformer}
Декодер Transformer стал предпочтительным декодером для задач прогнозирования последовательности, таких как машинный перевод. Он также широко применяется в задаче распознавания рукописного текста. Например, state-of-the-art решение согласно бенчмарку \href{https://paperswithcode.com/sota/handwritten-text-recognition-on-iam}{IAM} принадлежит на момент написания этого текста decoder-only transformer архитектуре \hyperlink{cite.Mas23}{[8]}. Минус данной архитектуры - это то, что она является более громоздкой. Она требует большего количества параметров и, соотвественно, данных для обучения \hyperlink{image5}{[Рис 5]}.

\begin{figure}
    \centering
    \includegraphics[scale=0.25]{./images/competition_all.png}
    \caption{\protect\hypertarget{image6}{Сравнение различных архитектур для задачи распознавания текста на различных бенмарках. \\ Взято из \protect\hyperlink{cite.Her21}{[20]}}.}
\end{figure}

\subsubsection{Connectionist Temporal Classification}
Декодер Connectionist Temporal Classification изначально использовался для распознавания речи, позже исследователи добились огромного успеха, применив его для задач распознавания текста. Далее представлен математический формализм временной классификации, а также ошибки, используемой в качестве метрики в данном исследовании.

Пусть S - это набор обучающих примеров, выбранных из фиксированного распределения $D_{\mathcal{X} \times \mathcal{Z}}$. Пространство входных данных $\mathcal{X} = (\mathbb{R}^m)^*$ представляет собой множество всех последовательностей из m-мерных векторов вещественных чисел. Целевое пространство $\mathcal{Z} = L^*$ представляет собой множество всех последовательностей из (конечного) алфавита $L$. В общем случае, мы обозначаем элементы $L^*$ как последовательности символов. Каждый пример в S состоит из пары последовательностей $(x, z)$. Целевая последовательность $z = (z_1, z_2, ..., z_U)$ не длиннее входной последовательности $x = (x_1, x_2, ..., x_T)$, то есть $U \leq T$. Поскольку входные и целевые последовательности обычно не имеют одинаковой длины, нет априорного способа их выравнивания.

Цель состоит в использовании $S$ для обучения временного классификатора $h : \mathcal{X} \rightarrow \mathcal{Z}$, который классифицирует ранее не виденные входные последовательности таким образом, чтобы минимизировать некоторую специфическую ошибку задачи.

Общепризнанным выбором является следующая ошибка: для заданного тестового набора $S_0 \subset D_{\mathcal{X} \times \mathcal{Z}}$, не пересекающегося с $S$, определяется коэффициент ошибок меток (label error rate, LER) временного классификатора $h$ как нормализованное расстояние Левенштейна между его предсказаниями и ответом на $S_0$, т.е.:
\begin{equation}
	LER(h, S_0) = \frac{1}{Z} \sum_{(x,z) \in S_0} ED(h(x), z)
\end{equation}
где $Z$ - общее количество целевых меток в $S_0$, а $ED(p, q)$ - расстояние Левенштейна между двумя последовательностями $p$ и $q$, т.е. минимальное количество вставок, замен и удалений, необходимых для преобразования $p$ в $q$.
Это естественная метрика для задач (таких как распознавание речи или рукописного текста), где целью является минимизация частоты ошибок транскрипции. Популярными метриками в задаче распознавания рукописного текста являются WER (word error rate) и CER (character error rate) - описанное выше расстояние Левенштейна на уровне слов и символов соответственно.

Также в данном исследовании будет использоваться метрика LabelAccuracy (Label может быть словом, строкой или фрагментом). Определяется она так:
\begin{equation}
	LabelAcc(h, S_0) = 100 * (1 - \frac{\sum_{(x,z) \in S_0} ED(h(x), z)}{\sum_{(x,z) \in S_0} |z|})
\end{equation}


Пусть мы имеем некоторую нейронную сеть $N_w : (\mathbb{R}^m)^T \rightarrow (\mathbb{R}^n)^T$. Пусть $y = N_w(x)$ - последовательность выходов сети, и обозначим $y_t^k$ активацию выходного узла $k$ в момент времени $t$ (последний слой softmax). Тогда $y_t^k$ интерпретируется как вероятность наблюдения символа $k$ в момент времени $t$, что определяет распределение над множеством $L_{0}^T$ длины $T$ последовательностей алфавита $L_0 = L \cup \{blank\}$. Тут мы добавляем в алфавит символ $blank$, обозначающий пропуск. Элементы $L_0^T$ обычно называют путями.

\begin{equation}
	p(\pi|x) = \prod_{t=1}^{T} y_t^{\pi_t}, \quad \forall \pi \in L_{0}^{T}
\end{equation}

Далее определим отображение $B: L_{0}^T \rightarrow L_{\leq T}$, где $L_{\leq T}$ - это множество последовательностей длиной менее или равной $T$ по оригинальному алфавиту $L$. Мы делаем это, просто удаляя все пробелы и повторяющиеся символы из путей. Интуитивно это соответствует выводу нового символа, когда сеть переходит от предсказания отсутствия символа к предсказанию символа, или от предсказания одного символа к другому. Наконец, используем $B$ для определения условной вероятности данного предсказания $I \in L_{\leq T}$ как сумму вероятностей всех путей, соответствующих ему:
\begin{equation}
	p(I|x) = \sum_{\pi \in B^{-1}(I)} p(\pi|x)
\end{equation}

Вывод классификатора должен быть наиболее вероятным предсказанием для входной последовательности:
\begin{equation}
	h(x) = \arg\max_{I \in L_{\leq T}} p(I|x)
\end{equation}

В данном исследовании поиск такого предсказания строится на предположении, что наиболее вероятный путь будет соответствовать наиболее вероятному предсказанию:
\begin{equation}
\begin{split}
	h(x) \approx B(\pi^{*}) \\
	\pi^{*} = \arg\max_{\pi \in L_0^T} p(\pi|x).
\end{split}
\end{equation}
Это легко найти, так как $\pi^{*}$ представляет собой просто конкатенацию самых вероятных символов на каждом этапе. Такой подход не гарантирует нахождение наиболее вероятного предсказания, но достаточно хорошо его приблежает. 

Модель обучается методом максимального правдоподобия. Для вычисления условных вероятностей $p(I|x)$ отдельных предсказаний используется динамическое программирование \hyperlink{cite.Gra06}{[2]}. Также включение явной языковой модели поверх логитов может значительно повысить точность \hyperlink{cite.Fuj17}{[24]}.
\paragraph{}
Во время выбора архитектуры учитывался собственный практический опыт, а также результаты исследований по данной теме. Например, в \hyperlink{cite.Her21}{[20]} проводится сравнительный анализ различных архитектур энкодеров / декодеров \hyperlink{image6}{[Рис 6]}. Было решено выбрать в качестве:
\begin{enumerate}
\item Сверточной нейронной сети: ResNet
\item Последовательного энкодера: TransformerEncoder
\item Декодера: Connectionist Temporal Classification
\end{enumerate}

\newpage %% Постановка задачи
    \section{Обзор существующих методов аугментации изображений для задачи распознавания рукописного текста}
\label{sec:Chapter2} \index{Chapter2}

\href{https://arxiv.org/pdf/2104.07787v2}{взять отсюда (3)}

\newpage %% Обзор существующих решений
    \section{Конфигурация}
\label{sec:Chapter3} \index{Chapter3}

\begin{figure}
    \centering
    Real \\
    \includegraphics[scale=0.45]{./images/data/Real.jpg} \\
    Synthetic
    \\
    \includegraphics[scale=3]{./images/data/Synthetic.jpg} \\
    SynForms
    \\
    \includegraphics[scale=0.4]{./images/data/SynForms.jpg}
    \caption{\protect\hypertarget{image11}{Примеры изображений из датасета.}}
\end{figure}

\subsection{Описание модели}
В качестве энкодера использовался TransformerEncoder с следующими параметрами:
\begin{enumerate}
\item количество слоев: $6$
\item nhead в multiheadattention): $4$
\item размерномть feedforward сети: $768$
\item размерность пространства признаков: $128$
\end{enumerate}

В качестве сверточной нейронной сети для извлечения визуальных признаков использовался ResNet \hyperlink{cite.Kai15}{[17]}. В качестве функции активации использовалась ReLU. Далее архитектура ResNet описывается чуть подробнее:

\subsubsection{ConvBN}
Слой ConvBN представляет из себя композицию свертки (conv) и батч-нормализации (BN). Параметры BN следующие:
\begin{enumerate}
\item eps: $10^{-12}$
\item momentum: $0.01$
\end{enumerate}
За filters будем обозначать количество фильтров свертки. За kernel размер ядра, padding и stride также есть соответсвующие параметры свертки.

\subsubsection{ResNetBlock}
Этот residual блок предстваляет из себя композицию трех ConvBN (с размерами ядра: $[1, 3, 1]$, padding: $[0, 1, 0]$). Функция активации: ReLU (как было сказано ранее).

\subsubsection{FeatureBlock}
Это композиция нескольких ResNetBlock. Также для всех FeatureBlock, кроме последнего, после ResNetBlock-ов есть еще ConvBN слой (kernel: $3$, padding: $1$). В некоторых случаях (кроме последних двух) данный блок завершается max-pool (kernel: $2$). После FeatureBlock для регуляризации использовался dropout \hyperlink{cite.Sut14}{[12]} (с вероятностью $0.1$).

Первый FeatureBlock отличается от остальных: он представляет из себя композицию двух ConvBN (kernel: $[3, 3]$, padding: $[1, 1]$) и max-pool (kernel: $2$) слоя.

\paragraph{}

ResNet преставляет из себя композицию нескольких FeatureBlock-ов (всего $5$, с количествами ResNetBlock: $[1, 2, 5, 3]$, первый FeatureBlock отличается от остальных). После FeatureBlock-ов идут два ConvBN и dropout (c вероятностью $0.2$). 


\subsection{Описание датасета}

Данные можно разделить на три категории:
\begin{enumerate}
\item Real: Данные, собранные с документов. Обычно это поля различных контрактов, форм, или поля на счетах
\item SynForms: Написанные людьми строки текста
\item Synthetic: Псевдорукописные шрифты после применения некоторых аугментаций
\end{enumerate}
Примеры изображений можно найти на \hyperlink{image11}{[Рис 11.]}. Для обучения брались данные из всех трех групп в соотношении $1:1:1$.

\begin{figure}
    \centering
    \includegraphics[scale=0.5]{./images/lr.png} 
    \includegraphics[scale=0.5]{./images/lr_log.png}
    \caption{\protect\hypertarget{image12}{Как менялся learning rate во время обучения.}}
\end{figure}

\subsection{Оптимизатор}
В качестве оптимизатора использовался AdamW \hyperlink{cite.Los17}{[29]} с параметрами:
\begin{enumerate}
\item betas: $(0.9, 0.999)$
\item weight decay: $0.01$
\end{enumerate}
Learning rate начинался с $0.0001$ и умножался на $0.95$ каждые две эпохи \hyperlink{image12}{[Рис 12.]}. Также во всех экспериментах размер батча был равен $64$. Использовалась точность \href{https://pytorch.org/blog/what-every-user-should-know-about-mixed-precision-training-in-pytorch/}{mixed precision в PyTorch}. Также, если не сказано иного, обучение ведется в течение $100$ эпох.

\subsection{Реализация mixup}
Для более эффективной работы, образцы $(x, y)$ и $(x_0, y_0)$ семплируются каждый раз внутри одного батча. Точнее говоря, если мы имеем батч $(x_1, \dots, x_N)$, то операция mixup выглядит следующим образом:

\begin{equation}
\begin{split}
y_{1:N} = shuffle(x_{1:N}) \\
\tilde{x}_{1:N} = \lambda x_{1:N} + (1 - \lambda) y_{1:N}
\end{split}
\end{equation}

Основываясь на результатах \hyperlink{cite.Bas19}{[16]}, по умолчанию (если не сказано иного) $\alpha = 0.5$, то есть $\lambda \sim Beta(0.5, 0.5)$. Каждый раз, во время обучения, mixup происходит лишь в одной позиции. Эта позиция выбирается случайно из набора.

\newpage %% Исследование и построение решения задачи
    \section{Эксперименты}
\label{sec:Chapter4} \index{Chapter4}

\begin{figure}
    \centering
    \includegraphics[scale=0.4]{./images/mixup_position/all/ValidLoss.png}
    \caption{\protect\hypertarget{image13}{Функция ошибки для всех рассматриваемых наборов позиций mixup.}}
\end{figure}

\subsection{Влияние положения Mixup}
Рассматривались положения mixup между FeatureBlock-ами, а также между последними двумя ConvBN. Точнее говоря, если описывать псевдокодом модель ResNet, рассматриваемые точки для mixup выглядят следующим образом:

\begin{lstlisting}
for i in range(5):
    x = mixup(x)                    # before_block_i
    x = self.blocks[i](x)

x = mixup(x)                        # after_blocks
x = self.final_conv1(x)
x = mixup(x)                        # after_final_conv1
x = self.final_conv2(x)
x = self.final_drop(x)
\end{lstlisting}

Основываясь на результатах \hyperlink{cite.Bas19}{[16]}, были проанализированы все варианты набора позиций для mixup, подходящие под следующие условия:
\begin{enumerate}
\item В каждом наборе присутствуют лишь 2 или 3 точки, где происходит mixup.
\item Позиции в наборе не идут последовательно друг за другом в порядке операций.
\end{enumerate}

Каждый набор позиций был проверен на части датасета размером в $N = 128'000$ образцов \hyperref[tab:mixup_position]{[Таблица 2]}. Как по всем метрикам, так и по функции ошибки, набор before-block0;before-block3 оказался наилучшим \hyperlink{image13}{[Рис 13.]} \hyperref[tab:mixup_position_metrics]{[Таблица 3]}.

\newpage
\subsection{Зависимость от количества данных}

\begin{table}[]
\centering
\begin{tabular}{||c c c c c||} 
 \hline
 type & ValidLoss & CharAcc & FragAcc & WordAcc \\ [0.5ex] 
 \hline\hline
 vanilla & 5.55 & 87.56 & 56.28 & 59.93\\ 
 \hline
 mixup & 4.85 & 88.05 & 56.87 & 60.93\\ [1ex] 
 \hline
\end{tabular}
\caption{Функция ошибки и метрики при $N = 640'000$ образцах}
\end{table}

\newpage
 %% Описание практической части:  Rj
    \section{Вывод}
\label{sec:Chapter5} \index{Chapter5}

В этой работе исследована стратегия обучения модели с attention-энкодером для задачи распознавания рукописного текста. Предложен метод применения manifold mixup к изображениям разного размера с использованием Connectionist Temporal Classification. Анализируется влияние manifold mixup на процесс обучения нейронной сети. Также исследованы различные аспекты метода, такие как влияние выбора позиции. 

Доказано, что этот метод действует как сильный регуляризатор. Продемонстрировано значительное улучшение результатов распознавания текста \hyperref[tab:mixup_size_max]{[Таблица 1]}. 

В результате экспериментов были обнаружены следующие свойства подхода mixup:
\begin{enumerate}
\item Mixup действует как сильный регуляризатор. В частности, функция ошибки на валидационной выборке достигает некоторой горизонтальной асимптоты, в то время как без mixup модель переобучается и функция ошибки начинает расти.
\item Mixup замедляет обучение на ранних эпохах. В частности, требуется некоторое количество эпох, чтобы функция ошибки и другие метрики начали превосходить результаты модели без mixup.
\item Mixup показывает улучшение всех метрик и функции ошибки при всех наборах позиций. Тем не менее, выбор набора влияет на результаты.
В связи с этим рекомендуется при использовании mixup проводить обучение в течение достаточного количества эпох (возможно дольше в сравнении с обучением без mixup), а также грамотно выбирать набор позиций (например, проводя эксперименты на небольшом количестве данных).
\end{enumerate}
\newpage
    \section*{Приложение}
\addcontentsline{toc}{section}{Приложение}
\label{sec:Apendix} \index{Apendix}

\subsection{Позиция mixup}

\begin{longtable}{cccc}
\centering
\includegraphics[scale=0.2]{./images/mixup_position/before_block_2;before_block_4_ValidLoss.png} & \includegraphics[scale=0.2]{./images/mixup_position/before_block_2;before_block_4_CharAcc.png} & \includegraphics[scale=0.2]{./images/mixup_position/before_block_2;before_block_4_FragAcc.png} & \includegraphics[scale=0.2]{./images/mixup_position/before_block_2;before_block_4_WordAcc.png}\\
\includegraphics[scale=0.2]{./images/mixup_position/after_blocks;before_block_3_ValidLoss.png} & \includegraphics[scale=0.2]{./images/mixup_position/after_blocks;before_block_3_CharAcc.png} & \includegraphics[scale=0.2]{./images/mixup_position/after_blocks;before_block_3_FragAcc.png} & \includegraphics[scale=0.2]{./images/mixup_position/after_blocks;before_block_3_WordAcc.png}\\
\includegraphics[scale=0.2]{./images/mixup_position/before_block_0;before_block_3_ValidLoss.png} & \includegraphics[scale=0.2]{./images/mixup_position/before_block_0;before_block_3_CharAcc.png} & \includegraphics[scale=0.2]{./images/mixup_position/before_block_0;before_block_3_FragAcc.png} & \includegraphics[scale=0.2]{./images/mixup_position/before_block_0;before_block_3_WordAcc.png}\\
\includegraphics[scale=0.2]{./images/mixup_position/after_blocks;before_block_1;before_block_3_ValidLoss.png} & \includegraphics[scale=0.2]{./images/mixup_position/after_blocks;before_block_1;before_block_3_CharAcc.png} & \includegraphics[scale=0.2]{./images/mixup_position/after_blocks;before_block_1;before_block_3_FragAcc.png} & \includegraphics[scale=0.2]{./images/mixup_position/after_blocks;before_block_1;before_block_3_WordAcc.png}\\
\includegraphics[scale=0.2]{./images/mixup_position/after_blocks;before_block_2_ValidLoss.png} & \includegraphics[scale=0.2]{./images/mixup_position/after_blocks;before_block_2_CharAcc.png} & \includegraphics[scale=0.2]{./images/mixup_position/after_blocks;before_block_2_FragAcc.png} & \includegraphics[scale=0.2]{./images/mixup_position/after_blocks;before_block_2_WordAcc.png}\\
\includegraphics[scale=0.2]{./images/mixup_position/after_blocks;before_block_0;before_block_3_ValidLoss.png} & \includegraphics[scale=0.2]{./images/mixup_position/after_blocks;before_block_0;before_block_3_CharAcc.png} & \includegraphics[scale=0.2]{./images/mixup_position/after_blocks;before_block_0;before_block_3_FragAcc.png} & \includegraphics[scale=0.2]{./images/mixup_position/after_blocks;before_block_0;before_block_3_WordAcc.png}\\
\includegraphics[scale=0.2]{./images/mixup_position/before_block_0;before_block_2_ValidLoss.png} & \includegraphics[scale=0.2]{./images/mixup_position/before_block_0;before_block_2_CharAcc.png} & \includegraphics[scale=0.2]{./images/mixup_position/before_block_0;before_block_2_FragAcc.png} & \includegraphics[scale=0.2]{./images/mixup_position/before_block_0;before_block_2_WordAcc.png}\\
\includegraphics[scale=0.2]{./images/mixup_position/before_block_1;before_block_4_ValidLoss.png} & \includegraphics[scale=0.2]{./images/mixup_position/before_block_1;before_block_4_CharAcc.png} & \includegraphics[scale=0.2]{./images/mixup_position/before_block_1;before_block_4_FragAcc.png} & \includegraphics[scale=0.2]{./images/mixup_position/before_block_1;before_block_4_WordAcc.png}\\
\includegraphics[scale=0.2]{./images/mixup_position/after_blocks;before_block_0_ValidLoss.png} & \includegraphics[scale=0.2]{./images/mixup_position/after_blocks;before_block_0_CharAcc.png} & \includegraphics[scale=0.2]{./images/mixup_position/after_blocks;before_block_0_FragAcc.png} & \includegraphics[scale=0.2]{./images/mixup_position/after_blocks;before_block_0_WordAcc.png}\\
\includegraphics[scale=0.2]{./images/mixup_position/after_final_conv1;before_block_1;before_block_4_ValidLoss.png} & \includegraphics[scale=0.2]{./images/mixup_position/after_final_conv1;before_block_1;before_block_4_CharAcc.png} & \includegraphics[scale=0.2]{./images/mixup_position/after_final_conv1;before_block_1;before_block_4_FragAcc.png} & \includegraphics[scale=0.2]{./images/mixup_position/after_final_conv1;before_block_1;before_block_4_WordAcc.png}\\
\includegraphics[scale=0.2]{./images/mixup_position/before_block_0;before_block_4_ValidLoss.png} & \includegraphics[scale=0.2]{./images/mixup_position/before_block_0;before_block_4_CharAcc.png} & \includegraphics[scale=0.2]{./images/mixup_position/before_block_0;before_block_4_FragAcc.png} & \includegraphics[scale=0.2]{./images/mixup_position/before_block_0;before_block_4_WordAcc.png}\\
\includegraphics[scale=0.2]{./images/mixup_position/after_blocks;before_block_1_ValidLoss.png} & \includegraphics[scale=0.2]{./images/mixup_position/after_blocks;before_block_1_CharAcc.png} & \includegraphics[scale=0.2]{./images/mixup_position/after_blocks;before_block_1_FragAcc.png} & \includegraphics[scale=0.2]{./images/mixup_position/after_blocks;before_block_1_WordAcc.png}\\
\includegraphics[scale=0.2]{./images/mixup_position/after_final_conv1;before_block_2;before_block_4_ValidLoss.png} & \includegraphics[scale=0.2]{./images/mixup_position/after_final_conv1;before_block_2;before_block_4_CharAcc.png} & \includegraphics[scale=0.2]{./images/mixup_position/after_final_conv1;before_block_2;before_block_4_FragAcc.png} & \includegraphics[scale=0.2]{./images/mixup_position/after_final_conv1;before_block_2;before_block_4_WordAcc.png}\\
\includegraphics[scale=0.2]{./images/mixup_position/after_final_conv1;before_block_3_ValidLoss.png} & \includegraphics[scale=0.2]{./images/mixup_position/after_final_conv1;before_block_3_CharAcc.png} & \includegraphics[scale=0.2]{./images/mixup_position/after_final_conv1;before_block_3_FragAcc.png} & \includegraphics[scale=0.2]{./images/mixup_position/after_final_conv1;before_block_3_WordAcc.png}\\
\includegraphics[scale=0.2]{./images/mixup_position/before_block_0;before_block_2;before_block_4_ValidLoss.png} & \includegraphics[scale=0.2]{./images/mixup_position/before_block_0;before_block_2;before_block_4_CharAcc.png} & \includegraphics[scale=0.2]{./images/mixup_position/before_block_0;before_block_2;before_block_4_FragAcc.png} & \includegraphics[scale=0.2]{./images/mixup_position/before_block_0;before_block_2;before_block_4_WordAcc.png}\\
\includegraphics[scale=0.2]{./images/mixup_position/after_final_conv1;before_block_0;before_block_4_ValidLoss.png} & \includegraphics[scale=0.2]{./images/mixup_position/after_final_conv1;before_block_0;before_block_4_CharAcc.png} & \includegraphics[scale=0.2]{./images/mixup_position/after_final_conv1;before_block_0;before_block_4_FragAcc.png} & \includegraphics[scale=0.2]{./images/mixup_position/after_final_conv1;before_block_0;before_block_4_WordAcc.png}\\
\includegraphics[scale=0.2]{./images/mixup_position/after_final_conv1;before_block_1_ValidLoss.png} & \includegraphics[scale=0.2]{./images/mixup_position/after_final_conv1;before_block_1_CharAcc.png} & \includegraphics[scale=0.2]{./images/mixup_position/after_final_conv1;before_block_1_FragAcc.png} & \includegraphics[scale=0.2]{./images/mixup_position/after_final_conv1;before_block_1_WordAcc.png}\\
\includegraphics[scale=0.2]{./images/mixup_position/after_final_conv1;before_block_0_ValidLoss.png} & \includegraphics[scale=0.2]{./images/mixup_position/after_final_conv1;before_block_0_CharAcc.png} & \includegraphics[scale=0.2]{./images/mixup_position/after_final_conv1;before_block_0_FragAcc.png} & \includegraphics[scale=0.2]{./images/mixup_position/after_final_conv1;before_block_0_WordAcc.png}\\
\includegraphics[scale=0.2]{./images/mixup_position/after_final_conv1;before_block_0;before_block_2_ValidLoss.png} & \includegraphics[scale=0.2]{./images/mixup_position/after_final_conv1;before_block_0;before_block_2_CharAcc.png} & \includegraphics[scale=0.2]{./images/mixup_position/after_final_conv1;before_block_0;before_block_2_FragAcc.png} & \includegraphics[scale=0.2]{./images/mixup_position/after_final_conv1;before_block_0;before_block_2_WordAcc.png}\\
\includegraphics[scale=0.2]{./images/mixup_position/_ValidLoss.png} & \includegraphics[scale=0.2]{./images/mixup_position/_CharAcc.png} & \includegraphics[scale=0.2]{./images/mixup_position/_FragAcc.png} & \includegraphics[scale=0.2]{./images/mixup_position/_WordAcc.png}\\
\includegraphics[scale=0.2]{./images/mixup_position/after_final_conv1;before_block_2_ValidLoss.png} & \includegraphics[scale=0.2]{./images/mixup_position/after_final_conv1;before_block_2_CharAcc.png} & \includegraphics[scale=0.2]{./images/mixup_position/after_final_conv1;before_block_2_FragAcc.png} & \includegraphics[scale=0.2]{./images/mixup_position/after_final_conv1;before_block_2_WordAcc.png}\\
\includegraphics[scale=0.2]{./images/mixup_position/after_final_conv1;before_block_1;before_block_3_ValidLoss.png} & \includegraphics[scale=0.2]{./images/mixup_position/after_final_conv1;before_block_1;before_block_3_CharAcc.png} & \includegraphics[scale=0.2]{./images/mixup_position/after_final_conv1;before_block_1;before_block_3_FragAcc.png} & \includegraphics[scale=0.2]{./images/mixup_position/after_final_conv1;before_block_1;before_block_3_WordAcc.png}\\
\includegraphics[scale=0.2]{./images/mixup_position/after_final_conv1;before_block_0;before_block_3_ValidLoss.png} & \includegraphics[scale=0.2]{./images/mixup_position/after_final_conv1;before_block_0;before_block_3_CharAcc.png} & \includegraphics[scale=0.2]{./images/mixup_position/after_final_conv1;before_block_0;before_block_3_FragAcc.png} & \includegraphics[scale=0.2]{./images/mixup_position/after_final_conv1;before_block_0;before_block_3_WordAcc.png}\\
\includegraphics[scale=0.2]{./images/mixup_position/after_blocks;before_block_0;before_block_2_ValidLoss.png} & \includegraphics[scale=0.2]{./images/mixup_position/after_blocks;before_block_0;before_block_2_CharAcc.png} & \includegraphics[scale=0.2]{./images/mixup_position/after_blocks;before_block_0;before_block_2_FragAcc.png} & \includegraphics[scale=0.2]{./images/mixup_position/after_blocks;before_block_0;before_block_2_WordAcc.png}\\
\includegraphics[scale=0.2]{./images/mixup_position/before_block_1;before_block_3_ValidLoss.png} & \includegraphics[scale=0.2]{./images/mixup_position/before_block_1;before_block_3_CharAcc.png} & \includegraphics[scale=0.2]{./images/mixup_position/before_block_1;before_block_3_FragAcc.png} & \includegraphics[scale=0.2]{./images/mixup_position/before_block_1;before_block_3_WordAcc.png}\\
\includegraphics[scale=0.2]{./images/mixup_position/after_final_conv1;before_block_4_ValidLoss.png} & \includegraphics[scale=0.2]{./images/mixup_position/after_final_conv1;before_block_4_CharAcc.png} & \includegraphics[scale=0.2]{./images/mixup_position/after_final_conv1;before_block_4_FragAcc.png} & \includegraphics[scale=0.2]{./images/mixup_position/after_final_conv1;before_block_4_WordAcc.png}\\
\caption{Графики функции ошибки и некоторых метрик для каждого варианта позиций mixup.}
\label{tab:mixup_position}
\end{longtable}


\begin{longtable}{cc}
\centering
    \includegraphics[scale=0.2]{./images/mixup_position/all/WordAcc.png} &
    \includegraphics[scale=0.2]{./images/mixup_position/all/CharAcc.png} \\
    \includegraphics[scale=0.2]{./images/mixup_position/all/FragAcc.png} &
    \includegraphics[scale=0.2]{./images/mixup_position/all/ValidLoss.png} \\
\caption{Функция ошибки и различные метрики для всех рассматриваемых наборов позиций mixup.}
\label{tab:mixup_position_metrics}
\end{longtable}

\subsection{Количество данных}

\begin{longtable}{cccc}
\centering
\includegraphics[scale=0.2]{./images/mixup_size/100_ValidLoss.png} & \includegraphics[scale=0.2]{./images/mixup_size/100_CharAcc.png} & \includegraphics[scale=0.2]{./images/mixup_size/100_FragAcc.png} & \includegraphics[scale=0.2]{./images/mixup_size/100_WordAcc.png}\\
\includegraphics[scale=0.2]{./images/mixup_size/500_ValidLoss.png} & \includegraphics[scale=0.2]{./images/mixup_size/500_CharAcc.png} & \includegraphics[scale=0.2]{./images/mixup_size/500_FragAcc.png} & \includegraphics[scale=0.2]{./images/mixup_size/500_WordAcc.png}\\
\includegraphics[scale=0.2]{./images/mixup_size/1000_ValidLoss.png} & \includegraphics[scale=0.2]{./images/mixup_size/1000_CharAcc.png} & \includegraphics[scale=0.2]{./images/mixup_size/1000_FragAcc.png} & \includegraphics[scale=0.2]{./images/mixup_size/1000_WordAcc.png}\\
\includegraphics[scale=0.2]{./images/mixup_size/4000_ValidLoss.png} & \includegraphics[scale=0.2]{./images/mixup_size/4000_CharAcc.png} & \includegraphics[scale=0.2]{./images/mixup_size/4000_FragAcc.png} & \includegraphics[scale=0.2]{./images/mixup_size/4000_WordAcc.png}\\
\includegraphics[scale=0.2]{./images/mixup_size/8000_ValidLoss.png} & \includegraphics[scale=0.2]{./images/mixup_size/8000_CharAcc.png} & \includegraphics[scale=0.2]{./images/mixup_size/8000_FragAcc.png} & \includegraphics[scale=0.2]{./images/mixup_size/8000_WordAcc.png}\\
\includegraphics[scale=0.2]{./images/mixup_size/10000_ValidLoss.png} & \includegraphics[scale=0.2]{./images/mixup_size/10000_CharAcc.png} & \includegraphics[scale=0.2]{./images/mixup_size/10000_FragAcc.png} & \includegraphics[scale=0.2]{./images/mixup_size/10000_WordAcc.png}\\
\caption{Сравнение ванильной модели и mixup на различных размерах датасета. }
\label{tab:mixup_size}
\end{longtable}

\begin{longtable}{cc}
\centering
\includegraphics[scale=0.4]{./images/mixup_size/all/ValidLoss.png} & \includegraphics[scale=0.4]{./images/mixup_size/all/CharAcc.png}\\
\includegraphics[scale=0.4]{./images/mixup_size/all/FragAcc.png} & \includegraphics[scale=0.4]{./images/mixup_size/all/WordAcc.png}\\
\caption{Сравнение метрик для ванильной модели и mixup в зависимости от размера датасета.}
\label{tab:mixup_size_all}
\end{longtable}


    %% НЕ ТРОГАЙТЕ!!!
    \nocite{*}
    \bibliography{references}

    %% в зависимости от надобности подключаем раздел "Приложение"
    % \newpage
    % \section*{Приложение}
\addcontentsline{toc}{section}{Приложение}
\label{sec:Apendix} \index{Apendix}

\subsection{Позиция mixup}

\begin{longtable}{cccc}
\centering
\includegraphics[scale=0.2]{./images/mixup_position/before_block_2;before_block_4_ValidLoss.png} & \includegraphics[scale=0.2]{./images/mixup_position/before_block_2;before_block_4_CharAcc.png} & \includegraphics[scale=0.2]{./images/mixup_position/before_block_2;before_block_4_FragAcc.png} & \includegraphics[scale=0.2]{./images/mixup_position/before_block_2;before_block_4_WordAcc.png}\\
\includegraphics[scale=0.2]{./images/mixup_position/after_blocks;before_block_3_ValidLoss.png} & \includegraphics[scale=0.2]{./images/mixup_position/after_blocks;before_block_3_CharAcc.png} & \includegraphics[scale=0.2]{./images/mixup_position/after_blocks;before_block_3_FragAcc.png} & \includegraphics[scale=0.2]{./images/mixup_position/after_blocks;before_block_3_WordAcc.png}\\
\includegraphics[scale=0.2]{./images/mixup_position/before_block_0;before_block_3_ValidLoss.png} & \includegraphics[scale=0.2]{./images/mixup_position/before_block_0;before_block_3_CharAcc.png} & \includegraphics[scale=0.2]{./images/mixup_position/before_block_0;before_block_3_FragAcc.png} & \includegraphics[scale=0.2]{./images/mixup_position/before_block_0;before_block_3_WordAcc.png}\\
\includegraphics[scale=0.2]{./images/mixup_position/after_blocks;before_block_1;before_block_3_ValidLoss.png} & \includegraphics[scale=0.2]{./images/mixup_position/after_blocks;before_block_1;before_block_3_CharAcc.png} & \includegraphics[scale=0.2]{./images/mixup_position/after_blocks;before_block_1;before_block_3_FragAcc.png} & \includegraphics[scale=0.2]{./images/mixup_position/after_blocks;before_block_1;before_block_3_WordAcc.png}\\
\includegraphics[scale=0.2]{./images/mixup_position/after_blocks;before_block_2_ValidLoss.png} & \includegraphics[scale=0.2]{./images/mixup_position/after_blocks;before_block_2_CharAcc.png} & \includegraphics[scale=0.2]{./images/mixup_position/after_blocks;before_block_2_FragAcc.png} & \includegraphics[scale=0.2]{./images/mixup_position/after_blocks;before_block_2_WordAcc.png}\\
\includegraphics[scale=0.2]{./images/mixup_position/after_blocks;before_block_0;before_block_3_ValidLoss.png} & \includegraphics[scale=0.2]{./images/mixup_position/after_blocks;before_block_0;before_block_3_CharAcc.png} & \includegraphics[scale=0.2]{./images/mixup_position/after_blocks;before_block_0;before_block_3_FragAcc.png} & \includegraphics[scale=0.2]{./images/mixup_position/after_blocks;before_block_0;before_block_3_WordAcc.png}\\
\includegraphics[scale=0.2]{./images/mixup_position/before_block_0;before_block_2_ValidLoss.png} & \includegraphics[scale=0.2]{./images/mixup_position/before_block_0;before_block_2_CharAcc.png} & \includegraphics[scale=0.2]{./images/mixup_position/before_block_0;before_block_2_FragAcc.png} & \includegraphics[scale=0.2]{./images/mixup_position/before_block_0;before_block_2_WordAcc.png}\\
\includegraphics[scale=0.2]{./images/mixup_position/before_block_1;before_block_4_ValidLoss.png} & \includegraphics[scale=0.2]{./images/mixup_position/before_block_1;before_block_4_CharAcc.png} & \includegraphics[scale=0.2]{./images/mixup_position/before_block_1;before_block_4_FragAcc.png} & \includegraphics[scale=0.2]{./images/mixup_position/before_block_1;before_block_4_WordAcc.png}\\
\includegraphics[scale=0.2]{./images/mixup_position/after_blocks;before_block_0_ValidLoss.png} & \includegraphics[scale=0.2]{./images/mixup_position/after_blocks;before_block_0_CharAcc.png} & \includegraphics[scale=0.2]{./images/mixup_position/after_blocks;before_block_0_FragAcc.png} & \includegraphics[scale=0.2]{./images/mixup_position/after_blocks;before_block_0_WordAcc.png}\\
\includegraphics[scale=0.2]{./images/mixup_position/after_final_conv1;before_block_1;before_block_4_ValidLoss.png} & \includegraphics[scale=0.2]{./images/mixup_position/after_final_conv1;before_block_1;before_block_4_CharAcc.png} & \includegraphics[scale=0.2]{./images/mixup_position/after_final_conv1;before_block_1;before_block_4_FragAcc.png} & \includegraphics[scale=0.2]{./images/mixup_position/after_final_conv1;before_block_1;before_block_4_WordAcc.png}\\
\includegraphics[scale=0.2]{./images/mixup_position/before_block_0;before_block_4_ValidLoss.png} & \includegraphics[scale=0.2]{./images/mixup_position/before_block_0;before_block_4_CharAcc.png} & \includegraphics[scale=0.2]{./images/mixup_position/before_block_0;before_block_4_FragAcc.png} & \includegraphics[scale=0.2]{./images/mixup_position/before_block_0;before_block_4_WordAcc.png}\\
\includegraphics[scale=0.2]{./images/mixup_position/after_blocks;before_block_1_ValidLoss.png} & \includegraphics[scale=0.2]{./images/mixup_position/after_blocks;before_block_1_CharAcc.png} & \includegraphics[scale=0.2]{./images/mixup_position/after_blocks;before_block_1_FragAcc.png} & \includegraphics[scale=0.2]{./images/mixup_position/after_blocks;before_block_1_WordAcc.png}\\
\includegraphics[scale=0.2]{./images/mixup_position/after_final_conv1;before_block_2;before_block_4_ValidLoss.png} & \includegraphics[scale=0.2]{./images/mixup_position/after_final_conv1;before_block_2;before_block_4_CharAcc.png} & \includegraphics[scale=0.2]{./images/mixup_position/after_final_conv1;before_block_2;before_block_4_FragAcc.png} & \includegraphics[scale=0.2]{./images/mixup_position/after_final_conv1;before_block_2;before_block_4_WordAcc.png}\\
\includegraphics[scale=0.2]{./images/mixup_position/after_final_conv1;before_block_3_ValidLoss.png} & \includegraphics[scale=0.2]{./images/mixup_position/after_final_conv1;before_block_3_CharAcc.png} & \includegraphics[scale=0.2]{./images/mixup_position/after_final_conv1;before_block_3_FragAcc.png} & \includegraphics[scale=0.2]{./images/mixup_position/after_final_conv1;before_block_3_WordAcc.png}\\
\includegraphics[scale=0.2]{./images/mixup_position/before_block_0;before_block_2;before_block_4_ValidLoss.png} & \includegraphics[scale=0.2]{./images/mixup_position/before_block_0;before_block_2;before_block_4_CharAcc.png} & \includegraphics[scale=0.2]{./images/mixup_position/before_block_0;before_block_2;before_block_4_FragAcc.png} & \includegraphics[scale=0.2]{./images/mixup_position/before_block_0;before_block_2;before_block_4_WordAcc.png}\\
\includegraphics[scale=0.2]{./images/mixup_position/after_final_conv1;before_block_0;before_block_4_ValidLoss.png} & \includegraphics[scale=0.2]{./images/mixup_position/after_final_conv1;before_block_0;before_block_4_CharAcc.png} & \includegraphics[scale=0.2]{./images/mixup_position/after_final_conv1;before_block_0;before_block_4_FragAcc.png} & \includegraphics[scale=0.2]{./images/mixup_position/after_final_conv1;before_block_0;before_block_4_WordAcc.png}\\
\includegraphics[scale=0.2]{./images/mixup_position/after_final_conv1;before_block_1_ValidLoss.png} & \includegraphics[scale=0.2]{./images/mixup_position/after_final_conv1;before_block_1_CharAcc.png} & \includegraphics[scale=0.2]{./images/mixup_position/after_final_conv1;before_block_1_FragAcc.png} & \includegraphics[scale=0.2]{./images/mixup_position/after_final_conv1;before_block_1_WordAcc.png}\\
\includegraphics[scale=0.2]{./images/mixup_position/after_final_conv1;before_block_0_ValidLoss.png} & \includegraphics[scale=0.2]{./images/mixup_position/after_final_conv1;before_block_0_CharAcc.png} & \includegraphics[scale=0.2]{./images/mixup_position/after_final_conv1;before_block_0_FragAcc.png} & \includegraphics[scale=0.2]{./images/mixup_position/after_final_conv1;before_block_0_WordAcc.png}\\
\includegraphics[scale=0.2]{./images/mixup_position/after_final_conv1;before_block_0;before_block_2_ValidLoss.png} & \includegraphics[scale=0.2]{./images/mixup_position/after_final_conv1;before_block_0;before_block_2_CharAcc.png} & \includegraphics[scale=0.2]{./images/mixup_position/after_final_conv1;before_block_0;before_block_2_FragAcc.png} & \includegraphics[scale=0.2]{./images/mixup_position/after_final_conv1;before_block_0;before_block_2_WordAcc.png}\\
\includegraphics[scale=0.2]{./images/mixup_position/_ValidLoss.png} & \includegraphics[scale=0.2]{./images/mixup_position/_CharAcc.png} & \includegraphics[scale=0.2]{./images/mixup_position/_FragAcc.png} & \includegraphics[scale=0.2]{./images/mixup_position/_WordAcc.png}\\
\includegraphics[scale=0.2]{./images/mixup_position/after_final_conv1;before_block_2_ValidLoss.png} & \includegraphics[scale=0.2]{./images/mixup_position/after_final_conv1;before_block_2_CharAcc.png} & \includegraphics[scale=0.2]{./images/mixup_position/after_final_conv1;before_block_2_FragAcc.png} & \includegraphics[scale=0.2]{./images/mixup_position/after_final_conv1;before_block_2_WordAcc.png}\\
\includegraphics[scale=0.2]{./images/mixup_position/after_final_conv1;before_block_1;before_block_3_ValidLoss.png} & \includegraphics[scale=0.2]{./images/mixup_position/after_final_conv1;before_block_1;before_block_3_CharAcc.png} & \includegraphics[scale=0.2]{./images/mixup_position/after_final_conv1;before_block_1;before_block_3_FragAcc.png} & \includegraphics[scale=0.2]{./images/mixup_position/after_final_conv1;before_block_1;before_block_3_WordAcc.png}\\
\includegraphics[scale=0.2]{./images/mixup_position/after_final_conv1;before_block_0;before_block_3_ValidLoss.png} & \includegraphics[scale=0.2]{./images/mixup_position/after_final_conv1;before_block_0;before_block_3_CharAcc.png} & \includegraphics[scale=0.2]{./images/mixup_position/after_final_conv1;before_block_0;before_block_3_FragAcc.png} & \includegraphics[scale=0.2]{./images/mixup_position/after_final_conv1;before_block_0;before_block_3_WordAcc.png}\\
\includegraphics[scale=0.2]{./images/mixup_position/after_blocks;before_block_0;before_block_2_ValidLoss.png} & \includegraphics[scale=0.2]{./images/mixup_position/after_blocks;before_block_0;before_block_2_CharAcc.png} & \includegraphics[scale=0.2]{./images/mixup_position/after_blocks;before_block_0;before_block_2_FragAcc.png} & \includegraphics[scale=0.2]{./images/mixup_position/after_blocks;before_block_0;before_block_2_WordAcc.png}\\
\includegraphics[scale=0.2]{./images/mixup_position/before_block_1;before_block_3_ValidLoss.png} & \includegraphics[scale=0.2]{./images/mixup_position/before_block_1;before_block_3_CharAcc.png} & \includegraphics[scale=0.2]{./images/mixup_position/before_block_1;before_block_3_FragAcc.png} & \includegraphics[scale=0.2]{./images/mixup_position/before_block_1;before_block_3_WordAcc.png}\\
\includegraphics[scale=0.2]{./images/mixup_position/after_final_conv1;before_block_4_ValidLoss.png} & \includegraphics[scale=0.2]{./images/mixup_position/after_final_conv1;before_block_4_CharAcc.png} & \includegraphics[scale=0.2]{./images/mixup_position/after_final_conv1;before_block_4_FragAcc.png} & \includegraphics[scale=0.2]{./images/mixup_position/after_final_conv1;before_block_4_WordAcc.png}\\
\caption{Графики функции ошибки и некоторых метрик для каждого варианта позиций mixup.}
\label{tab:mixup_position}
\end{longtable}


\begin{longtable}{cc}
\centering
    \includegraphics[scale=0.2]{./images/mixup_position/all/WordAcc.png} &
    \includegraphics[scale=0.2]{./images/mixup_position/all/CharAcc.png} \\
    \includegraphics[scale=0.2]{./images/mixup_position/all/FragAcc.png} &
    \includegraphics[scale=0.2]{./images/mixup_position/all/ValidLoss.png} \\
\caption{Функция ошибки и различные метрики для всех рассматриваемых наборов позиций mixup.}
\label{tab:mixup_position_metrics}
\end{longtable}

\subsection{Количество данных}

\begin{longtable}{cccc}
\centering
\includegraphics[scale=0.2]{./images/mixup_size/100_ValidLoss.png} & \includegraphics[scale=0.2]{./images/mixup_size/100_CharAcc.png} & \includegraphics[scale=0.2]{./images/mixup_size/100_FragAcc.png} & \includegraphics[scale=0.2]{./images/mixup_size/100_WordAcc.png}\\
\includegraphics[scale=0.2]{./images/mixup_size/500_ValidLoss.png} & \includegraphics[scale=0.2]{./images/mixup_size/500_CharAcc.png} & \includegraphics[scale=0.2]{./images/mixup_size/500_FragAcc.png} & \includegraphics[scale=0.2]{./images/mixup_size/500_WordAcc.png}\\
\includegraphics[scale=0.2]{./images/mixup_size/1000_ValidLoss.png} & \includegraphics[scale=0.2]{./images/mixup_size/1000_CharAcc.png} & \includegraphics[scale=0.2]{./images/mixup_size/1000_FragAcc.png} & \includegraphics[scale=0.2]{./images/mixup_size/1000_WordAcc.png}\\
\includegraphics[scale=0.2]{./images/mixup_size/4000_ValidLoss.png} & \includegraphics[scale=0.2]{./images/mixup_size/4000_CharAcc.png} & \includegraphics[scale=0.2]{./images/mixup_size/4000_FragAcc.png} & \includegraphics[scale=0.2]{./images/mixup_size/4000_WordAcc.png}\\
\includegraphics[scale=0.2]{./images/mixup_size/8000_ValidLoss.png} & \includegraphics[scale=0.2]{./images/mixup_size/8000_CharAcc.png} & \includegraphics[scale=0.2]{./images/mixup_size/8000_FragAcc.png} & \includegraphics[scale=0.2]{./images/mixup_size/8000_WordAcc.png}\\
\includegraphics[scale=0.2]{./images/mixup_size/10000_ValidLoss.png} & \includegraphics[scale=0.2]{./images/mixup_size/10000_CharAcc.png} & \includegraphics[scale=0.2]{./images/mixup_size/10000_FragAcc.png} & \includegraphics[scale=0.2]{./images/mixup_size/10000_WordAcc.png}\\
\caption{Сравнение ванильной модели и mixup на различных размерах датасета. }
\label{tab:mixup_size}
\end{longtable}

\begin{longtable}{cc}
\centering
\includegraphics[scale=0.4]{./images/mixup_size/all/ValidLoss.png} & \includegraphics[scale=0.4]{./images/mixup_size/all/CharAcc.png}\\
\includegraphics[scale=0.4]{./images/mixup_size/all/FragAcc.png} & \includegraphics[scale=0.4]{./images/mixup_size/all/WordAcc.png}\\
\caption{Сравнение метрик для ванильной модели и mixup в зависимости от размера датасета.}
\label{tab:mixup_size_all}
\end{longtable}

\end{document}
