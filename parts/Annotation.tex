\begin{abstract}

    \begin{center}
        \large{Аугментации в задаче распознавания рукописного текста} \\
    \large\textit{Курочкин Павел Сергеевич} \\[1 cm]

    В современных методах распознавания рукописного текста ключевым элементом является наличие обширного массива аннотированных данных. Однако, в случае ограниченности данных, эффективное применение методов аугментации становится критически важным для повышения точности моделей. Manifold Mixup \hyperlink{cite.Ver18}{[1]} - современный метод аугментации данных, он представляет собой подход, в котором объединяются два или более изображения или их признаковые карты для создания новых образцов данных. В данном исследовании предлагается метод адаптации Manifold Mixup для использования с подходами, основанными на Connectionist Temporal Classification \hyperlink{cite.Gra06}{[2]}, в контексте задачи распознавания рукописного текста. Проводится всестороннее исследование данного метода, анализируется его влияние на процесс обучения нейронной сети. Результаты исследования показывают значительное улучшение результатов распознавания текста при использовании Manifold Mixup.

    \vfill

    \textbf{Abstract} \\[1 cm]

    Augmentations in handwritten text recognition
    \end{center}

\end{abstract}
\newpage