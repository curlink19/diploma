\section{Введение}
\label{sec:Chapter0} \index{Chapter0}

Распознавание текста является важным этапом в большинстве приложений по анализу изображений документов. Оно позволяет автоматически получать доступ к информации, содержащейся на страницах. За последнее десятилетие произошло огромное улучшение систем распознавания рукописного текста, связанное с появлением новых подходов и архитектур. Совершенствование рекурентных сетей (в частности, одними из современных подходов являются 2D-LSTM \hyperlink{cite.Gra08}{[4]}, \hyperlink{cite.Bas14}{[5]} и 1D-LSTM \hyperlink{cite.Joa17}{[6]}, \hyperlink{cite.The17}{[7]}), а также появление архитектуры Transformer \hyperlink{cite.Vas17}{[3]} оказали значительное влияние на этот прогресс.
\subsection{Проблема}

Использование современных методов нейронных сетей помогает решить проблему высокой стилистической гетерогенности рукописного текста. Однако для таких мощных моделей с большим количеством параметров требуется значительное количество аннотированных изображений для обучения. Было предложено несколько методов, позволяющих уменьшить необходимость аннотированных данных при обучении систем распознавания текста.

Во-первых, обучающий набор можно расширить, добавив синтетические изображения. Это можно сделать, используя рукописные шрифты \hyperlink{cite.Mur03}{[9]} или создавая изображения текстовых строк на основе индивидуально извлеченных изображений реальных букв \hyperlink{cite.She16}{[10]}. Кроме того, для создания рукописных текстов можно применять генеративно-состязательные сети \hyperlink{cite.Alo19}{[11]}. 

Также, при ограниченном объеме данных, необходимо бороться с проблемой переобучения. Для этого было предложено множество подходов к регуляризации, таких как weight decay, dropout \hyperlink{cite.Sut14}{[12]}, batch normalization \hyperlink{cite.Iof15}{[21]} или раннее прекращение обучения, которые могут быть использованы в процессе обучения сети.

Наконец, для увеличения разнообразия обучающего набора, улучшения обобщающей способности модели и снижения риска переобучения широко применяются методы аугментации данных. Изображения могут быть отражены горизонтально и вертикально, подвергнуты поворотам и угловым трансформациям, изменены по размеру и масштабу. Также возможно добавление шума, обрезка изображений, цветовые трансформации и искажения в различных формах \hyperlink{cite.Cur17}{[13]}, \hyperlink{cite.Bas14}{[5]}. Подробнее методы аугментации для задачи распознавания рукописного текста описаны в \hyperref[sec:Chapter2]{Глава 3}.

В \hyperlink{cite.Cha11}{[14]} для создания новых образцов было предложено интерполировать признаки образцов одного класса. В \hyperlink{cite.Dev17}{[15]} было предложено использовать данный подход для признаков образцов на выходе промежуточного слоя нейронной сети. В \hyperlink{cite.Ver18}{[1]} объединяются обе эти идеи и предлагается смешивать целевую переменную и изображения из разных классов, или их промежуточные признаки на различных уровнях сети. Последний подход был назван Manifold Mixup. В \hyperlink{cite.Bas19}{[16]} представлен способ адаптации Manifold Mixup для работы с решениями, основанными на Connectionist Temporal Classification \hyperlink{cite.Gra06}{[2]}.

Цель данного исследования заключается в анализе методов аугментации изображений в контексте распознавания рукописного текста и более детальном рассмотрении различных аспектов метода Manifold Mixup. Выбор использования Manifold Mixup обоснован следующими уникальными характеристиками данного метода:
\begin{enumerate}
\item Его легкость в обобщении на широкий спектр задач компьютерного зрения, решаемых с помощью нейронных сетей.
\item Согласно результатам исследований \hyperlink{cite.Ver18}{[1]}, \hyperlink{cite.Bas19}{[16]}, данный метод значительно способствует качеству и обобщенности обученной модели.
\end{enumerate}
 
\subsection{Постановка задачи}
Рассматриваемые методы аугментации изображений, включая метод Manifold Mixup \hyperlink{cite.Ver18}{[1]}, исследуемый в данной работе, предназначены для преодоления следующих проблем, возникающих в задаче распознавания рукописного текста:
\begin{enumerate}
\item Разнообразие изображений рукописного текста из-за различий в стилях авторов и фонах документов.
\item Недостаток аннотированных данных, что затрудняет обобщение моделей.
\item Риск переобучения из-за упомянутых выше проблем и большого количества параметров в сети.
\end{enumerate}
В данном исследовании проводится анализ различных методов агументаций, решающих данные проблемы. А также исследуются следующие аспекты метода Manifold Mixup:
\begin{enumerate}
\item Методы формирования батчей из изображений различных длин и их влияние на стабильность процесса обучения.
\item Роль выбора промежуточного слоя в процессе обучения для различных архитектур.
\item Воздействие Manifold Mixup на процесс обучения в зависимости от объема имеющихся данных.
\end{enumerate}

В \hyperref[sec:Chapter1]{Главe 2} приводится обзор существующих архитектур для решения задачи распознавания рукописного текста. Также представлена архитектура, применяемая в данном исследовании, обосновывается выбор данной архитектуры.


\hyperref[sec:Chapter2]{Глава 3} посвящена обзору существующих методов аугментации изображений для задачи распознавания рукописного текста, включая Manifold Mixup. Оценивается эффективность их применения для решения вышеупомянутых задач.

В \hyperref[sec:Chapter3]{Главе 4} и \hyperref[sec:Chapter4]{Главе 5} представлено описание практической части: конфигурация и эксперименты соответсвенно.

И, наконец, в \hyperref[sec:Chapter5]{Главе 6} содержатся выводы из проведенного исследования.

\newpage