\section{Эксперименты}
\label{sec:Chapter4} \index{Chapter4}

\subsection{Влияние положения Mixup}
Рассматривались положения mixup между FeatureBlock-ами, а также между последними двумя ConvBN. Точнее говоря, если описывать псевдокодом модель ResNet, рассматриваемые точки для mixup выглядят следующим образом:

\begin{lstlisting}
for i in range(5):
    x = mixup(x)                    # before_block_i
    x = self.blocks[i](x)

x = mixup(x)                        # after_blocks
x = self.final_conv1(x)
x = mixup(x)                        # after_final_conv1
x = self.final_conv2(x)
x = self.final_drop(x)
\end{lstlisting}

Основываясь на результатах \hyperlink{cite.Bas19}{[16]}, были проанализированы все варианты позиций для mixup, подходящие под следующие условия:
\begin{enumerate}
\item В каждом наборе присутствуют лишь 2 или 3 точки, где происходит mixup.
\item Позиции в наборе не идут последовательно друг за другом в порядке операций.
\end{enumerate}

Каждый вариант позиций был проверен на части датасета размером в $N = 128'000$ образцов \hyperref[tab:mixup_position]{[Таблица 1]}.

\newpage
