\section{Вывод}
\label{sec:Chapter5} \index{Chapter5}

В этой работе исследована стратегия обучения модели с attention-энкодером для задачи распознавания рукописного текста. Предложен метод применения manifold mixup к изображениям разного размера с использованием Connectionist Temporal Classification. Анализируется влияние manifold mixup на процесс обучения нейронной сети. Также исследованы различные аспекты метода, такие как влияние выбора позиции. 

Доказано, что этот метод действует как сильный регуляризатор. Продемонстрировано значительное улучшение результатов распознавания текста \hyperref[tab:mixup_size_max]{[Таблица 1]}. 

В результате экспериментов были обнаружены следующие свойства подхода mixup:
\begin{enumerate}
\item Mixup действует как сильный регуляризатор. В частности, функция ошибки на валидационной выборке достигает некоторой горизонтальной асимптоты, в то время как без mixup модель переобучается и функция ошибки начинает расти.
\item Mixup замедляет обучение на ранних эпохах. В частности, требуется некоторое количество эпох, чтобы функция ошибки и другие метрики начали превосходить результаты модели без mixup.
\item Mixup показывает улучшение всех метрик и функции ошибки при всех наборах позиций. Тем не менее, выбор набора влияет на результаты.
В связи с этим рекомендуется при использовании mixup проводить обучение в течение достаточного количества эпох (возможно дольше в сравнении с обучением без mixup), а также грамотно выбирать набор позиций (например, проводя эксперименты на небольшом количестве данных).
\end{enumerate}
\newpage